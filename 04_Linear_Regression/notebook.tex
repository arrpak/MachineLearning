
% Default to the notebook output style

    


% Inherit from the specified cell style.




    
\documentclass[11pt]{article}

    
    
    \usepackage[T1]{fontenc}
    % Nicer default font (+ math font) than Computer Modern for most use cases
    \usepackage{mathpazo}

    % Basic figure setup, for now with no caption control since it's done
    % automatically by Pandoc (which extracts ![](path) syntax from Markdown).
    \usepackage{graphicx}
    % We will generate all images so they have a width \maxwidth. This means
    % that they will get their normal width if they fit onto the page, but
    % are scaled down if they would overflow the margins.
    \makeatletter
    \def\maxwidth{\ifdim\Gin@nat@width>\linewidth\linewidth
    \else\Gin@nat@width\fi}
    \makeatother
    \let\Oldincludegraphics\includegraphics
    % Set max figure width to be 80% of text width, for now hardcoded.
    \renewcommand{\includegraphics}[1]{\Oldincludegraphics[width=.8\maxwidth]{#1}}
    % Ensure that by default, figures have no caption (until we provide a
    % proper Figure object with a Caption API and a way to capture that
    % in the conversion process - todo).
    \usepackage{caption}
    \DeclareCaptionLabelFormat{nolabel}{}
    \captionsetup{labelformat=nolabel}

    \usepackage{adjustbox} % Used to constrain images to a maximum size 
    \usepackage{xcolor} % Allow colors to be defined
    \usepackage{enumerate} % Needed for markdown enumerations to work
    \usepackage{geometry} % Used to adjust the document margins
    \usepackage{amsmath} % Equations
    \usepackage{amssymb} % Equations
    \usepackage{textcomp} % defines textquotesingle
    % Hack from http://tex.stackexchange.com/a/47451/13684:
    \AtBeginDocument{%
        \def\PYZsq{\textquotesingle}% Upright quotes in Pygmentized code
    }
    \usepackage{upquote} % Upright quotes for verbatim code
    \usepackage{eurosym} % defines \euro
    \usepackage[mathletters]{ucs} % Extended unicode (utf-8) support
    \usepackage[utf8x]{inputenc} % Allow utf-8 characters in the tex document
    \usepackage{fancyvrb} % verbatim replacement that allows latex
    \usepackage{grffile} % extends the file name processing of package graphics 
                         % to support a larger range 
    % The hyperref package gives us a pdf with properly built
    % internal navigation ('pdf bookmarks' for the table of contents,
    % internal cross-reference links, web links for URLs, etc.)
    \usepackage{hyperref}
    \usepackage{longtable} % longtable support required by pandoc >1.10
    \usepackage{booktabs}  % table support for pandoc > 1.12.2
    \usepackage[inline]{enumitem} % IRkernel/repr support (it uses the enumerate* environment)
    \usepackage[normalem]{ulem} % ulem is needed to support strikethroughs (\sout)
                                % normalem makes italics be italics, not underlines
    

    
    
    % Colors for the hyperref package
    \definecolor{urlcolor}{rgb}{0,.145,.698}
    \definecolor{linkcolor}{rgb}{.71,0.21,0.01}
    \definecolor{citecolor}{rgb}{.12,.54,.11}

    % ANSI colors
    \definecolor{ansi-black}{HTML}{3E424D}
    \definecolor{ansi-black-intense}{HTML}{282C36}
    \definecolor{ansi-red}{HTML}{E75C58}
    \definecolor{ansi-red-intense}{HTML}{B22B31}
    \definecolor{ansi-green}{HTML}{00A250}
    \definecolor{ansi-green-intense}{HTML}{007427}
    \definecolor{ansi-yellow}{HTML}{DDB62B}
    \definecolor{ansi-yellow-intense}{HTML}{B27D12}
    \definecolor{ansi-blue}{HTML}{208FFB}
    \definecolor{ansi-blue-intense}{HTML}{0065CA}
    \definecolor{ansi-magenta}{HTML}{D160C4}
    \definecolor{ansi-magenta-intense}{HTML}{A03196}
    \definecolor{ansi-cyan}{HTML}{60C6C8}
    \definecolor{ansi-cyan-intense}{HTML}{258F8F}
    \definecolor{ansi-white}{HTML}{C5C1B4}
    \definecolor{ansi-white-intense}{HTML}{A1A6B2}

    % commands and environments needed by pandoc snippets
    % extracted from the output of `pandoc -s`
    \providecommand{\tightlist}{%
      \setlength{\itemsep}{0pt}\setlength{\parskip}{0pt}}
    \DefineVerbatimEnvironment{Highlighting}{Verbatim}{commandchars=\\\{\}}
    % Add ',fontsize=\small' for more characters per line
    \newenvironment{Shaded}{}{}
    \newcommand{\KeywordTok}[1]{\textcolor[rgb]{0.00,0.44,0.13}{\textbf{{#1}}}}
    \newcommand{\DataTypeTok}[1]{\textcolor[rgb]{0.56,0.13,0.00}{{#1}}}
    \newcommand{\DecValTok}[1]{\textcolor[rgb]{0.25,0.63,0.44}{{#1}}}
    \newcommand{\BaseNTok}[1]{\textcolor[rgb]{0.25,0.63,0.44}{{#1}}}
    \newcommand{\FloatTok}[1]{\textcolor[rgb]{0.25,0.63,0.44}{{#1}}}
    \newcommand{\CharTok}[1]{\textcolor[rgb]{0.25,0.44,0.63}{{#1}}}
    \newcommand{\StringTok}[1]{\textcolor[rgb]{0.25,0.44,0.63}{{#1}}}
    \newcommand{\CommentTok}[1]{\textcolor[rgb]{0.38,0.63,0.69}{\textit{{#1}}}}
    \newcommand{\OtherTok}[1]{\textcolor[rgb]{0.00,0.44,0.13}{{#1}}}
    \newcommand{\AlertTok}[1]{\textcolor[rgb]{1.00,0.00,0.00}{\textbf{{#1}}}}
    \newcommand{\FunctionTok}[1]{\textcolor[rgb]{0.02,0.16,0.49}{{#1}}}
    \newcommand{\RegionMarkerTok}[1]{{#1}}
    \newcommand{\ErrorTok}[1]{\textcolor[rgb]{1.00,0.00,0.00}{\textbf{{#1}}}}
    \newcommand{\NormalTok}[1]{{#1}}
    
    % Additional commands for more recent versions of Pandoc
    \newcommand{\ConstantTok}[1]{\textcolor[rgb]{0.53,0.00,0.00}{{#1}}}
    \newcommand{\SpecialCharTok}[1]{\textcolor[rgb]{0.25,0.44,0.63}{{#1}}}
    \newcommand{\VerbatimStringTok}[1]{\textcolor[rgb]{0.25,0.44,0.63}{{#1}}}
    \newcommand{\SpecialStringTok}[1]{\textcolor[rgb]{0.73,0.40,0.53}{{#1}}}
    \newcommand{\ImportTok}[1]{{#1}}
    \newcommand{\DocumentationTok}[1]{\textcolor[rgb]{0.73,0.13,0.13}{\textit{{#1}}}}
    \newcommand{\AnnotationTok}[1]{\textcolor[rgb]{0.38,0.63,0.69}{\textbf{\textit{{#1}}}}}
    \newcommand{\CommentVarTok}[1]{\textcolor[rgb]{0.38,0.63,0.69}{\textbf{\textit{{#1}}}}}
    \newcommand{\VariableTok}[1]{\textcolor[rgb]{0.10,0.09,0.49}{{#1}}}
    \newcommand{\ControlFlowTok}[1]{\textcolor[rgb]{0.00,0.44,0.13}{\textbf{{#1}}}}
    \newcommand{\OperatorTok}[1]{\textcolor[rgb]{0.40,0.40,0.40}{{#1}}}
    \newcommand{\BuiltInTok}[1]{{#1}}
    \newcommand{\ExtensionTok}[1]{{#1}}
    \newcommand{\PreprocessorTok}[1]{\textcolor[rgb]{0.74,0.48,0.00}{{#1}}}
    \newcommand{\AttributeTok}[1]{\textcolor[rgb]{0.49,0.56,0.16}{{#1}}}
    \newcommand{\InformationTok}[1]{\textcolor[rgb]{0.38,0.63,0.69}{\textbf{\textit{{#1}}}}}
    \newcommand{\WarningTok}[1]{\textcolor[rgb]{0.38,0.63,0.69}{\textbf{\textit{{#1}}}}}
    
    
    % Define a nice break command that doesn't care if a line doesn't already
    % exist.
    \def\br{\hspace*{\fill} \\* }
    % Math Jax compatability definitions
    \def\gt{>}
    \def\lt{<}
    % Document parameters
    \title{03\_Multiple\_LR\_Statsmodel}
    
    
    

    % Pygments definitions
    
\makeatletter
\def\PY@reset{\let\PY@it=\relax \let\PY@bf=\relax%
    \let\PY@ul=\relax \let\PY@tc=\relax%
    \let\PY@bc=\relax \let\PY@ff=\relax}
\def\PY@tok#1{\csname PY@tok@#1\endcsname}
\def\PY@toks#1+{\ifx\relax#1\empty\else%
    \PY@tok{#1}\expandafter\PY@toks\fi}
\def\PY@do#1{\PY@bc{\PY@tc{\PY@ul{%
    \PY@it{\PY@bf{\PY@ff{#1}}}}}}}
\def\PY#1#2{\PY@reset\PY@toks#1+\relax+\PY@do{#2}}

\expandafter\def\csname PY@tok@w\endcsname{\def\PY@tc##1{\textcolor[rgb]{0.73,0.73,0.73}{##1}}}
\expandafter\def\csname PY@tok@c\endcsname{\let\PY@it=\textit\def\PY@tc##1{\textcolor[rgb]{0.25,0.50,0.50}{##1}}}
\expandafter\def\csname PY@tok@cp\endcsname{\def\PY@tc##1{\textcolor[rgb]{0.74,0.48,0.00}{##1}}}
\expandafter\def\csname PY@tok@k\endcsname{\let\PY@bf=\textbf\def\PY@tc##1{\textcolor[rgb]{0.00,0.50,0.00}{##1}}}
\expandafter\def\csname PY@tok@kp\endcsname{\def\PY@tc##1{\textcolor[rgb]{0.00,0.50,0.00}{##1}}}
\expandafter\def\csname PY@tok@kt\endcsname{\def\PY@tc##1{\textcolor[rgb]{0.69,0.00,0.25}{##1}}}
\expandafter\def\csname PY@tok@o\endcsname{\def\PY@tc##1{\textcolor[rgb]{0.40,0.40,0.40}{##1}}}
\expandafter\def\csname PY@tok@ow\endcsname{\let\PY@bf=\textbf\def\PY@tc##1{\textcolor[rgb]{0.67,0.13,1.00}{##1}}}
\expandafter\def\csname PY@tok@nb\endcsname{\def\PY@tc##1{\textcolor[rgb]{0.00,0.50,0.00}{##1}}}
\expandafter\def\csname PY@tok@nf\endcsname{\def\PY@tc##1{\textcolor[rgb]{0.00,0.00,1.00}{##1}}}
\expandafter\def\csname PY@tok@nc\endcsname{\let\PY@bf=\textbf\def\PY@tc##1{\textcolor[rgb]{0.00,0.00,1.00}{##1}}}
\expandafter\def\csname PY@tok@nn\endcsname{\let\PY@bf=\textbf\def\PY@tc##1{\textcolor[rgb]{0.00,0.00,1.00}{##1}}}
\expandafter\def\csname PY@tok@ne\endcsname{\let\PY@bf=\textbf\def\PY@tc##1{\textcolor[rgb]{0.82,0.25,0.23}{##1}}}
\expandafter\def\csname PY@tok@nv\endcsname{\def\PY@tc##1{\textcolor[rgb]{0.10,0.09,0.49}{##1}}}
\expandafter\def\csname PY@tok@no\endcsname{\def\PY@tc##1{\textcolor[rgb]{0.53,0.00,0.00}{##1}}}
\expandafter\def\csname PY@tok@nl\endcsname{\def\PY@tc##1{\textcolor[rgb]{0.63,0.63,0.00}{##1}}}
\expandafter\def\csname PY@tok@ni\endcsname{\let\PY@bf=\textbf\def\PY@tc##1{\textcolor[rgb]{0.60,0.60,0.60}{##1}}}
\expandafter\def\csname PY@tok@na\endcsname{\def\PY@tc##1{\textcolor[rgb]{0.49,0.56,0.16}{##1}}}
\expandafter\def\csname PY@tok@nt\endcsname{\let\PY@bf=\textbf\def\PY@tc##1{\textcolor[rgb]{0.00,0.50,0.00}{##1}}}
\expandafter\def\csname PY@tok@nd\endcsname{\def\PY@tc##1{\textcolor[rgb]{0.67,0.13,1.00}{##1}}}
\expandafter\def\csname PY@tok@s\endcsname{\def\PY@tc##1{\textcolor[rgb]{0.73,0.13,0.13}{##1}}}
\expandafter\def\csname PY@tok@sd\endcsname{\let\PY@it=\textit\def\PY@tc##1{\textcolor[rgb]{0.73,0.13,0.13}{##1}}}
\expandafter\def\csname PY@tok@si\endcsname{\let\PY@bf=\textbf\def\PY@tc##1{\textcolor[rgb]{0.73,0.40,0.53}{##1}}}
\expandafter\def\csname PY@tok@se\endcsname{\let\PY@bf=\textbf\def\PY@tc##1{\textcolor[rgb]{0.73,0.40,0.13}{##1}}}
\expandafter\def\csname PY@tok@sr\endcsname{\def\PY@tc##1{\textcolor[rgb]{0.73,0.40,0.53}{##1}}}
\expandafter\def\csname PY@tok@ss\endcsname{\def\PY@tc##1{\textcolor[rgb]{0.10,0.09,0.49}{##1}}}
\expandafter\def\csname PY@tok@sx\endcsname{\def\PY@tc##1{\textcolor[rgb]{0.00,0.50,0.00}{##1}}}
\expandafter\def\csname PY@tok@m\endcsname{\def\PY@tc##1{\textcolor[rgb]{0.40,0.40,0.40}{##1}}}
\expandafter\def\csname PY@tok@gh\endcsname{\let\PY@bf=\textbf\def\PY@tc##1{\textcolor[rgb]{0.00,0.00,0.50}{##1}}}
\expandafter\def\csname PY@tok@gu\endcsname{\let\PY@bf=\textbf\def\PY@tc##1{\textcolor[rgb]{0.50,0.00,0.50}{##1}}}
\expandafter\def\csname PY@tok@gd\endcsname{\def\PY@tc##1{\textcolor[rgb]{0.63,0.00,0.00}{##1}}}
\expandafter\def\csname PY@tok@gi\endcsname{\def\PY@tc##1{\textcolor[rgb]{0.00,0.63,0.00}{##1}}}
\expandafter\def\csname PY@tok@gr\endcsname{\def\PY@tc##1{\textcolor[rgb]{1.00,0.00,0.00}{##1}}}
\expandafter\def\csname PY@tok@ge\endcsname{\let\PY@it=\textit}
\expandafter\def\csname PY@tok@gs\endcsname{\let\PY@bf=\textbf}
\expandafter\def\csname PY@tok@gp\endcsname{\let\PY@bf=\textbf\def\PY@tc##1{\textcolor[rgb]{0.00,0.00,0.50}{##1}}}
\expandafter\def\csname PY@tok@go\endcsname{\def\PY@tc##1{\textcolor[rgb]{0.53,0.53,0.53}{##1}}}
\expandafter\def\csname PY@tok@gt\endcsname{\def\PY@tc##1{\textcolor[rgb]{0.00,0.27,0.87}{##1}}}
\expandafter\def\csname PY@tok@err\endcsname{\def\PY@bc##1{\setlength{\fboxsep}{0pt}\fcolorbox[rgb]{1.00,0.00,0.00}{1,1,1}{\strut ##1}}}
\expandafter\def\csname PY@tok@kc\endcsname{\let\PY@bf=\textbf\def\PY@tc##1{\textcolor[rgb]{0.00,0.50,0.00}{##1}}}
\expandafter\def\csname PY@tok@kd\endcsname{\let\PY@bf=\textbf\def\PY@tc##1{\textcolor[rgb]{0.00,0.50,0.00}{##1}}}
\expandafter\def\csname PY@tok@kn\endcsname{\let\PY@bf=\textbf\def\PY@tc##1{\textcolor[rgb]{0.00,0.50,0.00}{##1}}}
\expandafter\def\csname PY@tok@kr\endcsname{\let\PY@bf=\textbf\def\PY@tc##1{\textcolor[rgb]{0.00,0.50,0.00}{##1}}}
\expandafter\def\csname PY@tok@bp\endcsname{\def\PY@tc##1{\textcolor[rgb]{0.00,0.50,0.00}{##1}}}
\expandafter\def\csname PY@tok@fm\endcsname{\def\PY@tc##1{\textcolor[rgb]{0.00,0.00,1.00}{##1}}}
\expandafter\def\csname PY@tok@vc\endcsname{\def\PY@tc##1{\textcolor[rgb]{0.10,0.09,0.49}{##1}}}
\expandafter\def\csname PY@tok@vg\endcsname{\def\PY@tc##1{\textcolor[rgb]{0.10,0.09,0.49}{##1}}}
\expandafter\def\csname PY@tok@vi\endcsname{\def\PY@tc##1{\textcolor[rgb]{0.10,0.09,0.49}{##1}}}
\expandafter\def\csname PY@tok@vm\endcsname{\def\PY@tc##1{\textcolor[rgb]{0.10,0.09,0.49}{##1}}}
\expandafter\def\csname PY@tok@sa\endcsname{\def\PY@tc##1{\textcolor[rgb]{0.73,0.13,0.13}{##1}}}
\expandafter\def\csname PY@tok@sb\endcsname{\def\PY@tc##1{\textcolor[rgb]{0.73,0.13,0.13}{##1}}}
\expandafter\def\csname PY@tok@sc\endcsname{\def\PY@tc##1{\textcolor[rgb]{0.73,0.13,0.13}{##1}}}
\expandafter\def\csname PY@tok@dl\endcsname{\def\PY@tc##1{\textcolor[rgb]{0.73,0.13,0.13}{##1}}}
\expandafter\def\csname PY@tok@s2\endcsname{\def\PY@tc##1{\textcolor[rgb]{0.73,0.13,0.13}{##1}}}
\expandafter\def\csname PY@tok@sh\endcsname{\def\PY@tc##1{\textcolor[rgb]{0.73,0.13,0.13}{##1}}}
\expandafter\def\csname PY@tok@s1\endcsname{\def\PY@tc##1{\textcolor[rgb]{0.73,0.13,0.13}{##1}}}
\expandafter\def\csname PY@tok@mb\endcsname{\def\PY@tc##1{\textcolor[rgb]{0.40,0.40,0.40}{##1}}}
\expandafter\def\csname PY@tok@mf\endcsname{\def\PY@tc##1{\textcolor[rgb]{0.40,0.40,0.40}{##1}}}
\expandafter\def\csname PY@tok@mh\endcsname{\def\PY@tc##1{\textcolor[rgb]{0.40,0.40,0.40}{##1}}}
\expandafter\def\csname PY@tok@mi\endcsname{\def\PY@tc##1{\textcolor[rgb]{0.40,0.40,0.40}{##1}}}
\expandafter\def\csname PY@tok@il\endcsname{\def\PY@tc##1{\textcolor[rgb]{0.40,0.40,0.40}{##1}}}
\expandafter\def\csname PY@tok@mo\endcsname{\def\PY@tc##1{\textcolor[rgb]{0.40,0.40,0.40}{##1}}}
\expandafter\def\csname PY@tok@ch\endcsname{\let\PY@it=\textit\def\PY@tc##1{\textcolor[rgb]{0.25,0.50,0.50}{##1}}}
\expandafter\def\csname PY@tok@cm\endcsname{\let\PY@it=\textit\def\PY@tc##1{\textcolor[rgb]{0.25,0.50,0.50}{##1}}}
\expandafter\def\csname PY@tok@cpf\endcsname{\let\PY@it=\textit\def\PY@tc##1{\textcolor[rgb]{0.25,0.50,0.50}{##1}}}
\expandafter\def\csname PY@tok@c1\endcsname{\let\PY@it=\textit\def\PY@tc##1{\textcolor[rgb]{0.25,0.50,0.50}{##1}}}
\expandafter\def\csname PY@tok@cs\endcsname{\let\PY@it=\textit\def\PY@tc##1{\textcolor[rgb]{0.25,0.50,0.50}{##1}}}

\def\PYZbs{\char`\\}
\def\PYZus{\char`\_}
\def\PYZob{\char`\{}
\def\PYZcb{\char`\}}
\def\PYZca{\char`\^}
\def\PYZam{\char`\&}
\def\PYZlt{\char`\<}
\def\PYZgt{\char`\>}
\def\PYZsh{\char`\#}
\def\PYZpc{\char`\%}
\def\PYZdl{\char`\$}
\def\PYZhy{\char`\-}
\def\PYZsq{\char`\'}
\def\PYZdq{\char`\"}
\def\PYZti{\char`\~}
% for compatibility with earlier versions
\def\PYZat{@}
\def\PYZlb{[}
\def\PYZrb{]}
\makeatother


    % Exact colors from NB
    \definecolor{incolor}{rgb}{0.0, 0.0, 0.5}
    \definecolor{outcolor}{rgb}{0.545, 0.0, 0.0}



    
    % Prevent overflowing lines due to hard-to-break entities
    \sloppy 
    % Setup hyperref package
    \hypersetup{
      breaklinks=true,  % so long urls are correctly broken across lines
      colorlinks=true,
      urlcolor=urlcolor,
      linkcolor=linkcolor,
      citecolor=citecolor,
      }
    % Slightly bigger margins than the latex defaults
    
    \geometry{verbose,tmargin=1in,bmargin=1in,lmargin=1in,rmargin=1in}
    
    

    \begin{document}
    
    
    \maketitle
    
    

    
    \section{Multiple Linear Regression
Model}\label{multiple-linear-regression-model}

Usando Statsmodel

    \begin{Verbatim}[commandchars=\\\{\}]
{\color{incolor}In [{\color{incolor}115}]:} \PY{k+kn}{import} \PY{n+nn}{pandas} \PY{k}{as} \PY{n+nn}{pd}
          \PY{k+kn}{import} \PY{n+nn}{numpy} \PY{k}{as} \PY{n+nn}{np}
          \PY{k+kn}{import} \PY{n+nn}{matplotlib}\PY{n+nn}{.}\PY{n+nn}{pyplot} \PY{k}{as} \PY{n+nn}{plt}
          \PY{o}{\PYZpc{}}\PY{k}{matplotlib} inline
\end{Verbatim}


    \begin{itemize}
\tightlist
\item
  Tendremos multiples betas en función de los multiples features
\item
  Cada beta tendrá un p-valor asociado
\item
  Si p-valor es muy pequeño significa que está muy relacionado con la
  predicción
\item
  Si p-valor es muy grande significa que no es un buen predictor
\end{itemize}

    \begin{Verbatim}[commandchars=\\\{\}]
{\color{incolor}In [{\color{incolor}116}]:} \PY{n}{data} \PY{o}{=} \PY{n}{pd}\PY{o}{.}\PY{n}{read\PYZus{}csv}\PY{p}{(}\PY{l+s+s1}{\PYZsq{}}\PY{l+s+s1}{../datasets/ads/Advertising.csv}\PY{l+s+s1}{\PYZsq{}}\PY{p}{)}
          \PY{n}{data}\PY{o}{.}\PY{n}{head}\PY{p}{(}\PY{p}{)}
\end{Verbatim}


\begin{Verbatim}[commandchars=\\\{\}]
{\color{outcolor}Out[{\color{outcolor}116}]:}       TV  Radio  Newspaper  Sales
          0  230.1   37.8       69.2   22.1
          1   44.5   39.3       45.1   10.4
          2   17.2   45.9       69.3    9.3
          3  151.5   41.3       58.5   18.5
          4  180.8   10.8       58.4   12.9
\end{Verbatim}
            
    \section{PosiblesModelos}\label{posiblesmodelos}

\(Modelos = { 2 }^{ k }\quad -\quad 1\) \#\#\# 7 Opciones

\begin{itemize}
\tightlist
\item
  Sales \textasciitilde{} TV
\item
  Sales \textasciitilde{} Newspaper
\item
  Sales \textasciitilde{} Radio
\item
  Sales \textasciitilde{} TV + Newspaper
\item
  Sales \textasciitilde{} TV + Radio
\item
  Sales \textasciitilde{} Newspaper + Radio
\item
  Sales \textasciitilde{} TV + Newspaper + Radio
\end{itemize}

    \section{Reglas para escoger modelo (hacia
atras)}\label{reglas-para-escoger-modelo-hacia-atras}

\begin{enumerate}
\def\labelenumi{\arabic{enumi}.}
\tightlist
\item
  Coger todas las variables (features)
\item
  Ver summary del modelo resultante
\item
  Eliminar features con p-valor alto \textasciitilde{}
  \textgreater{}0.05
\item
  Asegurarse que el R2 aumenta
\end{enumerate}

\section{Reglas para escoger modelo (hacia
adelante)}\label{reglas-para-escoger-modelo-hacia-adelante}

\begin{enumerate}
\def\labelenumi{\arabic{enumi}.}
\tightlist
\item
  Coger variables con menor suma de los cuadrados residuales
\item
  Ver R2 o R2 ajustado
\item
  Añadir variable
\item
  Ver si R2 o R2 ajustado aumenta
\end{enumerate}

    \subsubsection{Sales TV}\label{sales-tv}

    \begin{Verbatim}[commandchars=\\\{\}]
{\color{incolor}In [{\color{incolor}117}]:} \PY{k+kn}{import} \PY{n+nn}{statsmodels}\PY{n+nn}{.}\PY{n+nn}{formula}\PY{n+nn}{.}\PY{n+nn}{api} \PY{k}{as} \PY{n+nn}{smf}
\end{Verbatim}


    \begin{Verbatim}[commandchars=\\\{\}]
{\color{incolor}In [{\color{incolor}118}]:} \PY{n}{lm} \PY{o}{=} \PY{n}{smf}\PY{o}{.}\PY{n}{ols}\PY{p}{(}\PY{n}{formula}\PY{o}{=}\PY{l+s+s1}{\PYZsq{}}\PY{l+s+s1}{Sales\PYZti{}TV}\PY{l+s+s1}{\PYZsq{}}\PY{p}{,}\PY{n}{data}\PY{o}{=}\PY{n}{data}\PY{p}{)}\PY{o}{.}\PY{n}{fit}\PY{p}{(}\PY{p}{)}
\end{Verbatim}


    \begin{Verbatim}[commandchars=\\\{\}]
{\color{incolor}In [{\color{incolor}119}]:} \PY{n}{lm}\PY{o}{.}\PY{n}{params}
\end{Verbatim}


\begin{Verbatim}[commandchars=\\\{\}]
{\color{outcolor}Out[{\color{outcolor}119}]:} Intercept    7.032594
          TV           0.047537
          dtype: float64
\end{Verbatim}
            
    \begin{Verbatim}[commandchars=\\\{\}]
{\color{incolor}In [{\color{incolor}120}]:} \PY{n}{lm}\PY{o}{.}\PY{n}{pvalues}
\end{Verbatim}


\begin{Verbatim}[commandchars=\\\{\}]
{\color{outcolor}Out[{\color{outcolor}120}]:} Intercept    1.406300e-35
          TV           1.467390e-42
          dtype: float64
\end{Verbatim}
            
    \begin{Verbatim}[commandchars=\\\{\}]
{\color{incolor}In [{\color{incolor}121}]:} \PY{n}{lm}\PY{o}{.}\PY{n}{summary}\PY{p}{(}\PY{p}{)}
\end{Verbatim}


\begin{Verbatim}[commandchars=\\\{\}]
{\color{outcolor}Out[{\color{outcolor}121}]:} <class 'statsmodels.iolib.summary.Summary'>
          """
                                      OLS Regression Results                            
          ==============================================================================
          Dep. Variable:                  Sales   R-squared:                       0.612
          Model:                            OLS   Adj. R-squared:                  0.610
          Method:                 Least Squares   F-statistic:                     312.1
          Date:                Thu, 07 Mar 2019   Prob (F-statistic):           1.47e-42
          Time:                        23:11:57   Log-Likelihood:                -519.05
          No. Observations:                 200   AIC:                             1042.
          Df Residuals:                     198   BIC:                             1049.
          Df Model:                           1                                         
          Covariance Type:            nonrobust                                         
          ==============================================================================
                           coef    std err          t      P>|t|      [0.025      0.975]
          ------------------------------------------------------------------------------
          Intercept      7.0326      0.458     15.360      0.000       6.130       7.935
          TV             0.0475      0.003     17.668      0.000       0.042       0.053
          ==============================================================================
          Omnibus:                        0.531   Durbin-Watson:                   1.935
          Prob(Omnibus):                  0.767   Jarque-Bera (JB):                0.669
          Skew:                          -0.089   Prob(JB):                        0.716
          Kurtosis:                       2.779   Cond. No.                         338.
          ==============================================================================
          
          Warnings:
          [1] Standard Errors assume that the covariance matrix of the errors is correctly specified.
          """
\end{Verbatim}
            
    \begin{Verbatim}[commandchars=\\\{\}]
{\color{incolor}In [{\color{incolor}122}]:} \PY{n}{sales\PYZus{}pred} \PY{o}{=} \PY{n}{lm}\PY{o}{.}\PY{n}{predict}\PY{p}{(}\PY{n}{data}\PY{p}{[}\PY{l+s+s1}{\PYZsq{}}\PY{l+s+s1}{TV}\PY{l+s+s1}{\PYZsq{}}\PY{p}{]}\PY{p}{)}
\end{Verbatim}


    \begin{Verbatim}[commandchars=\\\{\}]
{\color{incolor}In [{\color{incolor}123}]:} \PY{n}{SSD} \PY{o}{=} \PY{n+nb}{sum}\PY{p}{(}\PY{p}{(}\PY{n}{data}\PY{p}{[}\PY{l+s+s1}{\PYZsq{}}\PY{l+s+s1}{Sales}\PY{l+s+s1}{\PYZsq{}}\PY{p}{]}\PY{o}{\PYZhy{}} \PY{n}{sales\PYZus{}pred}\PY{p}{)}\PY{o}{*}\PY{o}{*}\PY{l+m+mi}{2}\PY{p}{)}
          \PY{n}{SSD}
\end{Verbatim}


\begin{Verbatim}[commandchars=\\\{\}]
{\color{outcolor}Out[{\color{outcolor}123}]:} 2102.5305831313503
\end{Verbatim}
            
    \begin{Verbatim}[commandchars=\\\{\}]
{\color{incolor}In [{\color{incolor}124}]:} \PY{n}{RSE} \PY{o}{=} \PY{n}{np}\PY{o}{.}\PY{n}{sqrt}\PY{p}{(}\PY{n}{SSD}\PY{o}{/}\PY{p}{(}\PY{n+nb}{len}\PY{p}{(}\PY{n}{data}\PY{p}{)}\PY{o}{\PYZhy{}}\PY{l+m+mi}{2}\PY{p}{)}\PY{p}{)}
          \PY{n}{RSE}
\end{Verbatim}


\begin{Verbatim}[commandchars=\\\{\}]
{\color{outcolor}Out[{\color{outcolor}124}]:} 3.258656368650462
\end{Verbatim}
            
    \begin{Verbatim}[commandchars=\\\{\}]
{\color{incolor}In [{\color{incolor}125}]:} \PY{n}{error} \PY{o}{=} \PY{n}{RSE}\PY{o}{/}\PY{n}{np}\PY{o}{.}\PY{n}{mean}\PY{p}{(}\PY{n}{data}\PY{p}{[}\PY{l+s+s1}{\PYZsq{}}\PY{l+s+s1}{Sales}\PY{l+s+s1}{\PYZsq{}}\PY{p}{]}\PY{p}{)}
          \PY{n}{error}
\end{Verbatim}


\begin{Verbatim}[commandchars=\\\{\}]
{\color{outcolor}Out[{\color{outcolor}125}]:} 0.23238768897489473
\end{Verbatim}
            
    \subsubsection{Sales TV Newspaper}\label{sales-tv-newspaper}

    \begin{Verbatim}[commandchars=\\\{\}]
{\color{incolor}In [{\color{incolor}126}]:} \PY{n}{lm2} \PY{o}{=} \PY{n}{smf}\PY{o}{.}\PY{n}{ols}\PY{p}{(}\PY{n}{formula}\PY{o}{=}\PY{l+s+s1}{\PYZsq{}}\PY{l+s+s1}{Sales\PYZti{}TV+Newspaper}\PY{l+s+s1}{\PYZsq{}}\PY{p}{,}\PY{n}{data}\PY{o}{=}\PY{n}{data}\PY{p}{)}\PY{o}{.}\PY{n}{fit}\PY{p}{(}\PY{p}{)}
\end{Verbatim}


    \begin{Verbatim}[commandchars=\\\{\}]
{\color{incolor}In [{\color{incolor}127}]:} \PY{n}{lm2}\PY{o}{.}\PY{n}{params}
\end{Verbatim}


\begin{Verbatim}[commandchars=\\\{\}]
{\color{outcolor}Out[{\color{outcolor}127}]:} Intercept    5.774948
          TV           0.046901
          Newspaper    0.044219
          dtype: float64
\end{Verbatim}
            
    \begin{Verbatim}[commandchars=\\\{\}]
{\color{incolor}In [{\color{incolor}128}]:} \PY{n}{lm2}\PY{o}{.}\PY{n}{pvalues}
\end{Verbatim}


\begin{Verbatim}[commandchars=\\\{\}]
{\color{outcolor}Out[{\color{outcolor}128}]:} Intercept    3.145860e-22
          TV           5.507584e-44
          Newspaper    2.217084e-05
          dtype: float64
\end{Verbatim}
            
    \begin{Verbatim}[commandchars=\\\{\}]
{\color{incolor}In [{\color{incolor}129}]:} \PY{c+c1}{\PYZsh{} Vemos mejoras en R2 y R2 adjusted de 0.612/0.610 a 0.646/0.642}
          \PY{c+c1}{\PYZsh{} También se observa una bajada en el AIC que significa un mejor modelo de 1042 a 1026}
          
          \PY{n}{lm2}\PY{o}{.}\PY{n}{summary}\PY{p}{(}\PY{p}{)}
\end{Verbatim}


\begin{Verbatim}[commandchars=\\\{\}]
{\color{outcolor}Out[{\color{outcolor}129}]:} <class 'statsmodels.iolib.summary.Summary'>
          """
                                      OLS Regression Results                            
          ==============================================================================
          Dep. Variable:                  Sales   R-squared:                       0.646
          Model:                            OLS   Adj. R-squared:                  0.642
          Method:                 Least Squares   F-statistic:                     179.6
          Date:                Thu, 07 Mar 2019   Prob (F-statistic):           3.95e-45
          Time:                        23:11:58   Log-Likelihood:                -509.89
          No. Observations:                 200   AIC:                             1026.
          Df Residuals:                     197   BIC:                             1036.
          Df Model:                           2                                         
          Covariance Type:            nonrobust                                         
          ==============================================================================
                           coef    std err          t      P>|t|      [0.025      0.975]
          ------------------------------------------------------------------------------
          Intercept      5.7749      0.525     10.993      0.000       4.739       6.811
          TV             0.0469      0.003     18.173      0.000       0.042       0.052
          Newspaper      0.0442      0.010      4.346      0.000       0.024       0.064
          ==============================================================================
          Omnibus:                        0.658   Durbin-Watson:                   1.969
          Prob(Omnibus):                  0.720   Jarque-Bera (JB):                0.415
          Skew:                          -0.093   Prob(JB):                        0.813
          Kurtosis:                       3.122   Cond. No.                         410.
          ==============================================================================
          
          Warnings:
          [1] Standard Errors assume that the covariance matrix of the errors is correctly specified.
          """
\end{Verbatim}
            
    \begin{Verbatim}[commandchars=\\\{\}]
{\color{incolor}In [{\color{incolor}130}]:} \PY{n}{sales\PYZus{}pred} \PY{o}{=} \PY{n}{lm2}\PY{o}{.}\PY{n}{predict}\PY{p}{(}\PY{n}{data}\PY{p}{[}\PY{p}{[}\PY{l+s+s1}{\PYZsq{}}\PY{l+s+s1}{TV}\PY{l+s+s1}{\PYZsq{}}\PY{p}{,}\PY{l+s+s1}{\PYZsq{}}\PY{l+s+s1}{Newspaper}\PY{l+s+s1}{\PYZsq{}}\PY{p}{]}\PY{p}{]}\PY{p}{)}
\end{Verbatim}


    \begin{Verbatim}[commandchars=\\\{\}]
{\color{incolor}In [{\color{incolor}131}]:} \PY{c+c1}{\PYZsh{} Hemos conseguido bajar SSD/ESS de 2102.53 a 1918.56}
          
          \PY{n}{SSD} \PY{o}{=} \PY{n+nb}{sum}\PY{p}{(}\PY{p}{(}\PY{n}{data}\PY{p}{[}\PY{l+s+s1}{\PYZsq{}}\PY{l+s+s1}{Sales}\PY{l+s+s1}{\PYZsq{}}\PY{p}{]}\PY{o}{\PYZhy{}} \PY{n}{sales\PYZus{}pred}\PY{p}{)}\PY{o}{*}\PY{o}{*}\PY{l+m+mi}{2}\PY{p}{)}
          \PY{n}{SSD}
\end{Verbatim}


\begin{Verbatim}[commandchars=\\\{\}]
{\color{outcolor}Out[{\color{outcolor}131}]:} 1918.5618118968275
\end{Verbatim}
            
    \begin{Verbatim}[commandchars=\\\{\}]
{\color{incolor}In [{\color{incolor}132}]:} \PY{c+c1}{\PYZsh{} Hemos bajado de 3.25 a 3.12!}
          
          \PY{n}{RSE} \PY{o}{=} \PY{n}{np}\PY{o}{.}\PY{n}{sqrt}\PY{p}{(}\PY{n}{SSD}\PY{o}{/}\PY{p}{(}\PY{n+nb}{len}\PY{p}{(}\PY{n}{data}\PY{p}{)}\PY{o}{\PYZhy{}}\PY{l+m+mi}{3}\PY{p}{)}\PY{p}{)}
          \PY{n}{RSE}
\end{Verbatim}


\begin{Verbatim}[commandchars=\\\{\}]
{\color{outcolor}Out[{\color{outcolor}132}]:} 3.1207198602528856
\end{Verbatim}
            
    \begin{Verbatim}[commandchars=\\\{\}]
{\color{incolor}In [{\color{incolor}133}]:} \PY{c+c1}{\PYZsh{} Ha bajado de 0.23 a 0.22}
          
          \PY{n}{error} \PY{o}{=} \PY{n}{RSE}\PY{o}{/}\PY{n}{np}\PY{o}{.}\PY{n}{mean}\PY{p}{(}\PY{n}{data}\PY{p}{[}\PY{l+s+s1}{\PYZsq{}}\PY{l+s+s1}{Sales}\PY{l+s+s1}{\PYZsq{}}\PY{p}{]}\PY{p}{)}
          \PY{n}{error}
\end{Verbatim}


\begin{Verbatim}[commandchars=\\\{\}]
{\color{outcolor}Out[{\color{outcolor}133}]:} 0.2225508903728212
\end{Verbatim}
            
    \textbf{\emph{Conclusión: Añandir Newspaper al modelo aporta una pequeña
mejora}}

    \subsubsection{Sales TV Radio}\label{sales-tv-radio}

    \begin{Verbatim}[commandchars=\\\{\}]
{\color{incolor}In [{\color{incolor}134}]:} \PY{n}{lm3} \PY{o}{=} \PY{n}{smf}\PY{o}{.}\PY{n}{ols}\PY{p}{(}\PY{n}{formula}\PY{o}{=}\PY{l+s+s1}{\PYZsq{}}\PY{l+s+s1}{Sales\PYZti{}TV+Radio}\PY{l+s+s1}{\PYZsq{}}\PY{p}{,}\PY{n}{data}\PY{o}{=}\PY{n}{data}\PY{p}{)}\PY{o}{.}\PY{n}{fit}\PY{p}{(}\PY{p}{)}
\end{Verbatim}


    \begin{Verbatim}[commandchars=\\\{\}]
{\color{incolor}In [{\color{incolor}135}]:} \PY{n}{lm3}\PY{o}{.}\PY{n}{params}
\end{Verbatim}


\begin{Verbatim}[commandchars=\\\{\}]
{\color{outcolor}Out[{\color{outcolor}135}]:} Intercept    2.921100
          TV           0.045755
          Radio        0.187994
          dtype: float64
\end{Verbatim}
            
    \begin{Verbatim}[commandchars=\\\{\}]
{\color{incolor}In [{\color{incolor}136}]:} \PY{n}{lm3}\PY{o}{.}\PY{n}{pvalues}
\end{Verbatim}


\begin{Verbatim}[commandchars=\\\{\}]
{\color{outcolor}Out[{\color{outcolor}136}]:} Intercept    4.565557e-19
          TV           5.436980e-82
          Radio        9.776972e-59
          dtype: float64
\end{Verbatim}
            
    \begin{Verbatim}[commandchars=\\\{\}]
{\color{incolor}In [{\color{incolor}137}]:} \PY{c+c1}{\PYZsh{} Vemos mejoras en R2 y R2 adjusted de 0.646/0.642 a 0.897/0.896}
          \PY{c+c1}{\PYZsh{} También se observa una bajada en el AIC que significa un mejor modelo de 1026 a 778.4}
          
          \PY{n}{lm3}\PY{o}{.}\PY{n}{summary}\PY{p}{(}\PY{p}{)}
\end{Verbatim}


\begin{Verbatim}[commandchars=\\\{\}]
{\color{outcolor}Out[{\color{outcolor}137}]:} <class 'statsmodels.iolib.summary.Summary'>
          """
                                      OLS Regression Results                            
          ==============================================================================
          Dep. Variable:                  Sales   R-squared:                       0.897
          Model:                            OLS   Adj. R-squared:                  0.896
          Method:                 Least Squares   F-statistic:                     859.6
          Date:                Thu, 07 Mar 2019   Prob (F-statistic):           4.83e-98
          Time:                        23:12:00   Log-Likelihood:                -386.20
          No. Observations:                 200   AIC:                             778.4
          Df Residuals:                     197   BIC:                             788.3
          Df Model:                           2                                         
          Covariance Type:            nonrobust                                         
          ==============================================================================
                           coef    std err          t      P>|t|      [0.025      0.975]
          ------------------------------------------------------------------------------
          Intercept      2.9211      0.294      9.919      0.000       2.340       3.502
          TV             0.0458      0.001     32.909      0.000       0.043       0.048
          Radio          0.1880      0.008     23.382      0.000       0.172       0.204
          ==============================================================================
          Omnibus:                       60.022   Durbin-Watson:                   2.081
          Prob(Omnibus):                  0.000   Jarque-Bera (JB):              148.679
          Skew:                          -1.323   Prob(JB):                     5.19e-33
          Kurtosis:                       6.292   Cond. No.                         425.
          ==============================================================================
          
          Warnings:
          [1] Standard Errors assume that the covariance matrix of the errors is correctly specified.
          """
\end{Verbatim}
            
    \begin{Verbatim}[commandchars=\\\{\}]
{\color{incolor}In [{\color{incolor}138}]:} \PY{n}{sales\PYZus{}pred} \PY{o}{=} \PY{n}{lm3}\PY{o}{.}\PY{n}{predict}\PY{p}{(}\PY{n}{data}\PY{p}{[}\PY{p}{[}\PY{l+s+s1}{\PYZsq{}}\PY{l+s+s1}{TV}\PY{l+s+s1}{\PYZsq{}}\PY{p}{,}\PY{l+s+s1}{\PYZsq{}}\PY{l+s+s1}{Radio}\PY{l+s+s1}{\PYZsq{}}\PY{p}{]}\PY{p}{]}\PY{p}{)}
\end{Verbatim}


    \begin{Verbatim}[commandchars=\\\{\}]
{\color{incolor}In [{\color{incolor}139}]:} \PY{c+c1}{\PYZsh{} Hemos conseguido bajar SSD(ESS) de 1918.56 a 556.91}
          
          \PY{n}{SSD} \PY{o}{=} \PY{n+nb}{sum}\PY{p}{(}\PY{p}{(}\PY{n}{data}\PY{p}{[}\PY{l+s+s1}{\PYZsq{}}\PY{l+s+s1}{Sales}\PY{l+s+s1}{\PYZsq{}}\PY{p}{]}\PY{o}{\PYZhy{}} \PY{n}{sales\PYZus{}pred}\PY{p}{)}\PY{o}{*}\PY{o}{*}\PY{l+m+mi}{2}\PY{p}{)}
          \PY{n}{SSD}
\end{Verbatim}


\begin{Verbatim}[commandchars=\\\{\}]
{\color{outcolor}Out[{\color{outcolor}139}]:} 556.9139800676184
\end{Verbatim}
            
    \begin{Verbatim}[commandchars=\\\{\}]
{\color{incolor}In [{\color{incolor}140}]:} \PY{c+c1}{\PYZsh{} Hemos bajado de 3.12 a 1.68}
          
          \PY{n}{RSE} \PY{o}{=} \PY{n}{np}\PY{o}{.}\PY{n}{sqrt}\PY{p}{(}\PY{n}{SSD}\PY{o}{/}\PY{p}{(}\PY{n+nb}{len}\PY{p}{(}\PY{n}{data}\PY{p}{)}\PY{o}{\PYZhy{}}\PY{l+m+mi}{3}\PY{p}{)}\PY{p}{)}
          \PY{n}{RSE}
\end{Verbatim}


\begin{Verbatim}[commandchars=\\\{\}]
{\color{outcolor}Out[{\color{outcolor}140}]:} 1.681360912508001
\end{Verbatim}
            
    \begin{Verbatim}[commandchars=\\\{\}]
{\color{incolor}In [{\color{incolor}141}]:} \PY{c+c1}{\PYZsh{} Ha bajado de 0.22 a 0.12}
          
          \PY{n}{error} \PY{o}{=} \PY{n}{RSE}\PY{o}{/}\PY{n}{np}\PY{o}{.}\PY{n}{mean}\PY{p}{(}\PY{n}{data}\PY{p}{[}\PY{l+s+s1}{\PYZsq{}}\PY{l+s+s1}{Sales}\PY{l+s+s1}{\PYZsq{}}\PY{p}{]}\PY{p}{)}
          \PY{n}{error}
\end{Verbatim}


\begin{Verbatim}[commandchars=\\\{\}]
{\color{outcolor}Out[{\color{outcolor}141}]:} 0.11990450436855059
\end{Verbatim}
            
    \textbf{\emph{Conclusión: Añandir Radio al modelo aporta una GRAN
mejora}}

    \subsubsection{Sales TV Radio Newspaper}\label{sales-tv-radio-newspaper}

    \begin{Verbatim}[commandchars=\\\{\}]
{\color{incolor}In [{\color{incolor}142}]:} \PY{n}{lm4} \PY{o}{=} \PY{n}{smf}\PY{o}{.}\PY{n}{ols}\PY{p}{(}\PY{n}{formula}\PY{o}{=}\PY{l+s+s1}{\PYZsq{}}\PY{l+s+s1}{Sales\PYZti{}TV+Radio+Newspaper}\PY{l+s+s1}{\PYZsq{}}\PY{p}{,}\PY{n}{data}\PY{o}{=}\PY{n}{data}\PY{p}{)}\PY{o}{.}\PY{n}{fit}\PY{p}{(}\PY{p}{)}
\end{Verbatim}


    \begin{Verbatim}[commandchars=\\\{\}]
{\color{incolor}In [{\color{incolor}143}]:} \PY{n}{lm4}\PY{o}{.}\PY{n}{params}
\end{Verbatim}


\begin{Verbatim}[commandchars=\\\{\}]
{\color{outcolor}Out[{\color{outcolor}143}]:} Intercept    2.938889
          TV           0.045765
          Radio        0.188530
          Newspaper   -0.001037
          dtype: float64
\end{Verbatim}
            
    \begin{Verbatim}[commandchars=\\\{\}]
{\color{incolor}In [{\color{incolor}144}]:} \PY{n}{lm4}\PY{o}{.}\PY{n}{pvalues}
\end{Verbatim}


\begin{Verbatim}[commandchars=\\\{\}]
{\color{outcolor}Out[{\color{outcolor}144}]:} Intercept    1.267295e-17
          TV           1.509960e-81
          Radio        1.505339e-54
          Newspaper    8.599151e-01
          dtype: float64
\end{Verbatim}
            
    \begin{Verbatim}[commandchars=\\\{\}]
{\color{incolor}In [{\color{incolor}145}]:} \PY{c+c1}{\PYZsh{} R2 y R2 adjusted de 0.897/0.896 a 0.897/0.896 NO MEJORA}
          \PY{c+c1}{\PYZsh{} También se observa una SUBIDA en el AIC que significa un PEOR modelo de 778.4 a 780.4}
          \PY{c+c1}{\PYZsh{} P valor de Newspaper es 0.860 MUY ALTO, indicador que es una variable que deberiamos quitar}
          \PY{c+c1}{\PYZsh{} Intervalo de Confienaza de Newspaper entre \PYZhy{}0.013 y 0.011 COGE EL CERO, indicador que va a tener muy poca relevancia en la predicción}
          
          \PY{n}{lm4}\PY{o}{.}\PY{n}{summary}\PY{p}{(}\PY{p}{)}
\end{Verbatim}


\begin{Verbatim}[commandchars=\\\{\}]
{\color{outcolor}Out[{\color{outcolor}145}]:} <class 'statsmodels.iolib.summary.Summary'>
          """
                                      OLS Regression Results                            
          ==============================================================================
          Dep. Variable:                  Sales   R-squared:                       0.897
          Model:                            OLS   Adj. R-squared:                  0.896
          Method:                 Least Squares   F-statistic:                     570.3
          Date:                Thu, 07 Mar 2019   Prob (F-statistic):           1.58e-96
          Time:                        23:12:01   Log-Likelihood:                -386.18
          No. Observations:                 200   AIC:                             780.4
          Df Residuals:                     196   BIC:                             793.6
          Df Model:                           3                                         
          Covariance Type:            nonrobust                                         
          ==============================================================================
                           coef    std err          t      P>|t|      [0.025      0.975]
          ------------------------------------------------------------------------------
          Intercept      2.9389      0.312      9.422      0.000       2.324       3.554
          TV             0.0458      0.001     32.809      0.000       0.043       0.049
          Radio          0.1885      0.009     21.893      0.000       0.172       0.206
          Newspaper     -0.0010      0.006     -0.177      0.860      -0.013       0.011
          ==============================================================================
          Omnibus:                       60.414   Durbin-Watson:                   2.084
          Prob(Omnibus):                  0.000   Jarque-Bera (JB):              151.241
          Skew:                          -1.327   Prob(JB):                     1.44e-33
          Kurtosis:                       6.332   Cond. No.                         454.
          ==============================================================================
          
          Warnings:
          [1] Standard Errors assume that the covariance matrix of the errors is correctly specified.
          """
\end{Verbatim}
            
    \begin{Verbatim}[commandchars=\\\{\}]
{\color{incolor}In [{\color{incolor}150}]:} \PY{n}{sales\PYZus{}pred} \PY{o}{=} \PY{n}{lm4}\PY{o}{.}\PY{n}{predict}\PY{p}{(}\PY{n}{data}\PY{p}{[}\PY{p}{[}\PY{l+s+s1}{\PYZsq{}}\PY{l+s+s1}{TV}\PY{l+s+s1}{\PYZsq{}}\PY{p}{,}\PY{l+s+s1}{\PYZsq{}}\PY{l+s+s1}{Radio}\PY{l+s+s1}{\PYZsq{}}\PY{p}{,}\PY{l+s+s1}{\PYZsq{}}\PY{l+s+s1}{Newspaper}\PY{l+s+s1}{\PYZsq{}}\PY{p}{]}\PY{p}{]}\PY{p}{)}
\end{Verbatim}


    \begin{Verbatim}[commandchars=\\\{\}]
{\color{incolor}In [{\color{incolor}147}]:} \PY{c+c1}{\PYZsh{} Hemos conseguido bajar SSD(ESS) de 556.91 a 556.82 MUY POCA MEJORAAA}
          
          \PY{n}{SSD} \PY{o}{=} \PY{n+nb}{sum}\PY{p}{(}\PY{p}{(}\PY{n}{data}\PY{p}{[}\PY{l+s+s1}{\PYZsq{}}\PY{l+s+s1}{Sales}\PY{l+s+s1}{\PYZsq{}}\PY{p}{]}\PY{o}{\PYZhy{}} \PY{n}{sales\PYZus{}pred}\PY{p}{)}\PY{o}{*}\PY{o}{*}\PY{l+m+mi}{2}\PY{p}{)}
          \PY{n}{SSD}
\end{Verbatim}


\begin{Verbatim}[commandchars=\\\{\}]
{\color{outcolor}Out[{\color{outcolor}147}]:} 556.8252629021871
\end{Verbatim}
            
    \begin{Verbatim}[commandchars=\\\{\}]
{\color{incolor}In [{\color{incolor}148}]:} \PY{c+c1}{\PYZsh{} Hemos SUBIDO de 1.681360 a 1.6855 PEOR RSE!!!}
          
          \PY{n}{RSE} \PY{o}{=} \PY{n}{np}\PY{o}{.}\PY{n}{sqrt}\PY{p}{(}\PY{n}{SSD}\PY{o}{/}\PY{p}{(}\PY{n+nb}{len}\PY{p}{(}\PY{n}{data}\PY{p}{)}\PY{o}{\PYZhy{}}\PY{l+m+mi}{4}\PY{p}{)}\PY{p}{)}
          \PY{n}{RSE}
\end{Verbatim}


\begin{Verbatim}[commandchars=\\\{\}]
{\color{outcolor}Out[{\color{outcolor}148}]:} 1.6855103734147439
\end{Verbatim}
            
    \begin{Verbatim}[commandchars=\\\{\}]
{\color{incolor}In [{\color{incolor}149}]:} \PY{c+c1}{\PYZsh{} Ha SUBIDO de 0.12 a 0.1202 PEOR MODELO}
          
          \PY{n}{error} \PY{o}{=} \PY{n}{RSE}\PY{o}{/}\PY{n}{np}\PY{o}{.}\PY{n}{mean}\PY{p}{(}\PY{n}{data}\PY{p}{[}\PY{l+s+s1}{\PYZsq{}}\PY{l+s+s1}{Sales}\PY{l+s+s1}{\PYZsq{}}\PY{p}{]}\PY{p}{)}
          \PY{n}{error}
\end{Verbatim}


\begin{Verbatim}[commandchars=\\\{\}]
{\color{outcolor}Out[{\color{outcolor}149}]:} 0.12020041885646236
\end{Verbatim}
            
    \textbf{\emph{Conclusión: Añandir Newspaper al modelo que ya tenía TV y
Radio NO aporta mejora}}

\textbf{\emph{ELIMINAR NEWSPAPER DEL MODELO}}

    \emph{¿Existe alguna teoría para explicar que si incluyo el periodico
empeoro el modelo?}

    \section{MULTICOLINEALIDAD}\label{multicolinealidad}

    \begin{Verbatim}[commandchars=\\\{\}]
{\color{incolor}In [{\color{incolor}151}]:} \PY{k+kn}{import} \PY{n+nn}{seaborn} \PY{k}{as} \PY{n+nn}{sns}
          \PY{n}{corr} \PY{o}{=} \PY{n}{data}\PY{o}{.}\PY{n}{corr}\PY{p}{(}\PY{p}{)}
          \PY{n}{sns}\PY{o}{.}\PY{n}{heatmap}\PY{p}{(}\PY{n}{corr}\PY{p}{,} 
                      \PY{n}{xticklabels}\PY{o}{=}\PY{n}{corr}\PY{o}{.}\PY{n}{columns}\PY{o}{.}\PY{n}{values}\PY{p}{,}
                      \PY{n}{yticklabels}\PY{o}{=}\PY{n}{corr}\PY{o}{.}\PY{n}{columns}\PY{o}{.}\PY{n}{values}\PY{p}{)}\PY{p}{;}
\end{Verbatim}


    \begin{center}
    \adjustimage{max size={0.9\linewidth}{0.9\paperheight}}{output_48_0.png}
    \end{center}
    { \hspace*{\fill} \\}
    
    \begin{Verbatim}[commandchars=\\\{\}]
{\color{incolor}In [{\color{incolor}152}]:} \PY{n}{data}\PY{o}{.}\PY{n}{corr}\PY{p}{(}\PY{p}{)}
\end{Verbatim}


\begin{Verbatim}[commandchars=\\\{\}]
{\color{outcolor}Out[{\color{outcolor}152}]:}                  TV     Radio  Newspaper     Sales
          TV         1.000000  0.054809   0.056648  0.782224
          Radio      0.054809  1.000000   0.354104  0.576223
          Newspaper  0.056648  0.354104   1.000000  0.228299
          Sales      0.782224  0.576223   0.228299  1.000000
\end{Verbatim}
            
    \emph{Hay una cierta correlación entre RADIO y NEWSPAPER}

    \subsubsection{VIF Variance Inflation
Factor}\label{vif-variance-inflation-factor}

Para ver si existe multicolinealidad

\begin{itemize}
\tightlist
\item
  VIF = 1 NADA DE CORRELACIÓN
\item
  1 \textless{} VIF \textless{} 5 CORRELACIÓN MODERADA (TODAVÍA PUEDEN
  FORMAR PARTE DEL MODELO)
\item
  VIF \textgreater{} 5 VARIABLES ALTAMENTE CORRELACIONADAS (ELIMINAR DEL
  MODELO)
\end{itemize}

    \begin{Verbatim}[commandchars=\\\{\}]
{\color{incolor}In [{\color{incolor}153}]:} \PY{c+c1}{\PYZsh{} Newspaper \PYZti{} TV + Radio : VIF = 1/(1\PYZhy{}R\PYZca{}2)    R\PYZca{}2 es el coef de determinación}
          \PY{n}{lm\PYZus{}n} \PY{o}{=} \PY{n}{smf}\PY{o}{.}\PY{n}{ols}\PY{p}{(}\PY{n}{formula}\PY{o}{=}\PY{l+s+s1}{\PYZsq{}}\PY{l+s+s1}{Newspaper\PYZti{}TV+Radio}\PY{l+s+s1}{\PYZsq{}}\PY{p}{,}\PY{n}{data}\PY{o}{=}\PY{n}{data}\PY{p}{)}\PY{o}{.}\PY{n}{fit}\PY{p}{(}\PY{p}{)}
          \PY{n}{rsquared\PYZus{}n} \PY{o}{=} \PY{n}{lm\PYZus{}n}\PY{o}{.}\PY{n}{rsquared}
          \PY{n}{VIF} \PY{o}{=} \PY{l+m+mi}{1}\PY{o}{/}\PY{p}{(}\PY{l+m+mi}{1}\PY{o}{\PYZhy{}}\PY{n}{rsquared\PYZus{}n}\PY{p}{)}
          \PY{n}{VIF}
\end{Verbatim}


\begin{Verbatim}[commandchars=\\\{\}]
{\color{outcolor}Out[{\color{outcolor}153}]:} 1.1451873787239288
\end{Verbatim}
            
    \begin{Verbatim}[commandchars=\\\{\}]
{\color{incolor}In [{\color{incolor}154}]:} \PY{c+c1}{\PYZsh{} TV \PYZti{} Radio + Newspaper : VIF = 1/(1\PYZhy{}R\PYZca{}2)    R\PYZca{}2 es el coef de determinación   }
          \PY{n}{lm\PYZus{}tv} \PY{o}{=} \PY{n}{smf}\PY{o}{.}\PY{n}{ols}\PY{p}{(}\PY{n}{formula}\PY{o}{=}\PY{l+s+s1}{\PYZsq{}}\PY{l+s+s1}{TV\PYZti{}Newspaper+Radio}\PY{l+s+s1}{\PYZsq{}}\PY{p}{,}\PY{n}{data}\PY{o}{=}\PY{n}{data}\PY{p}{)}\PY{o}{.}\PY{n}{fit}\PY{p}{(}\PY{p}{)}
          \PY{n}{rsquared\PYZus{}tv} \PY{o}{=} \PY{n}{lm\PYZus{}tv}\PY{o}{.}\PY{n}{rsquared}
          \PY{n}{VIF} \PY{o}{=} \PY{l+m+mi}{1}\PY{o}{/}\PY{p}{(}\PY{l+m+mi}{1}\PY{o}{\PYZhy{}}\PY{n}{rsquared\PYZus{}tv}\PY{p}{)}
          \PY{n}{VIF}
\end{Verbatim}


\begin{Verbatim}[commandchars=\\\{\}]
{\color{outcolor}Out[{\color{outcolor}154}]:} 1.00461078493965
\end{Verbatim}
            
    \begin{Verbatim}[commandchars=\\\{\}]
{\color{incolor}In [{\color{incolor}156}]:} \PY{c+c1}{\PYZsh{} Radio \PYZti{} TV + Newspaper : VIF = 1/(1\PYZhy{}R\PYZca{}2)    R\PYZca{}2 es el coef de determinación}
          \PY{n}{lm\PYZus{}r} \PY{o}{=} \PY{n}{smf}\PY{o}{.}\PY{n}{ols}\PY{p}{(}\PY{n}{formula}\PY{o}{=}\PY{l+s+s1}{\PYZsq{}}\PY{l+s+s1}{Radio\PYZti{}Newspaper+TV}\PY{l+s+s1}{\PYZsq{}}\PY{p}{,}\PY{n}{data}\PY{o}{=}\PY{n}{data}\PY{p}{)}\PY{o}{.}\PY{n}{fit}\PY{p}{(}\PY{p}{)}
          \PY{n}{rsquared\PYZus{}r} \PY{o}{=} \PY{n}{lm\PYZus{}r}\PY{o}{.}\PY{n}{rsquared}
          \PY{n}{VIF} \PY{o}{=} \PY{l+m+mi}{1}\PY{o}{/}\PY{p}{(}\PY{l+m+mi}{1}\PY{o}{\PYZhy{}}\PY{n}{rsquared\PYZus{}r}\PY{p}{)}
          \PY{n}{VIF}
\end{Verbatim}


\begin{Verbatim}[commandchars=\\\{\}]
{\color{outcolor}Out[{\color{outcolor}156}]:} 1.1449519171055351
\end{Verbatim}
            
    \textbf{\emph{Eliminamos Newspaper puesto que su VIF es mayor y ya
habiamos visto que no mejoraba el modelo creado con TV y Radio}}

    \begin{Verbatim}[commandchars=\\\{\}]
{\color{incolor}In [{\color{incolor}172}]:} \PY{c+c1}{\PYZsh{} Running RFE with the output number of the variable equal to 9}
          \PY{n}{X\PYZus{}train} \PY{o}{=} \PY{n}{data}\PY{p}{[}\PY{p}{[}\PY{l+s+s1}{\PYZsq{}}\PY{l+s+s1}{TV}\PY{l+s+s1}{\PYZsq{}}\PY{p}{,}\PY{l+s+s1}{\PYZsq{}}\PY{l+s+s1}{Radio}\PY{l+s+s1}{\PYZsq{}}\PY{p}{,}\PY{l+s+s1}{\PYZsq{}}\PY{l+s+s1}{Newspaper}\PY{l+s+s1}{\PYZsq{}}\PY{p}{]}\PY{p}{]}
          \PY{n}{y\PYZus{}train} \PY{o}{=} \PY{n}{data}\PY{p}{[}\PY{l+s+s1}{\PYZsq{}}\PY{l+s+s1}{Sales}\PY{l+s+s1}{\PYZsq{}}\PY{p}{]}
          \PY{k+kn}{from} \PY{n+nn}{sklearn}\PY{n+nn}{.}\PY{n+nn}{linear\PYZus{}model} \PY{k}{import} \PY{n}{LinearRegression}
          \PY{k+kn}{from} \PY{n+nn}{sklearn}\PY{n+nn}{.}\PY{n+nn}{feature\PYZus{}selection} \PY{k}{import} \PY{n}{RFE}
          \PY{n}{lm} \PY{o}{=} \PY{n}{LinearRegression}\PY{p}{(}\PY{p}{)}
          \PY{n}{rfe} \PY{o}{=} \PY{n}{RFE}\PY{p}{(}\PY{n}{lm}\PY{p}{,} \PY{l+m+mi}{2}\PY{p}{)}             \PY{c+c1}{\PYZsh{} running RFE}
          \PY{n}{rfe} \PY{o}{=} \PY{n}{rfe}\PY{o}{.}\PY{n}{fit}\PY{p}{(}\PY{n}{X\PYZus{}train}\PY{p}{,} \PY{n}{y\PYZus{}train}\PY{p}{)}
          \PY{n+nb}{print}\PY{p}{(}\PY{n}{rfe}\PY{o}{.}\PY{n}{support\PYZus{}}\PY{p}{)}           \PY{c+c1}{\PYZsh{} Printing the boolean results}
          \PY{n+nb}{print}\PY{p}{(}\PY{n}{rfe}\PY{o}{.}\PY{n}{ranking\PYZus{}}\PY{p}{)}  
\end{Verbatim}


    \begin{Verbatim}[commandchars=\\\{\}]
[ True  True False]
[1 1 2]

    \end{Verbatim}

    \begin{Verbatim}[commandchars=\\\{\}]
{\color{incolor}In [{\color{incolor}166}]:} \PY{k+kn}{from} \PY{n+nn}{sklearn}\PY{n+nn}{.}\PY{n+nn}{feature\PYZus{}selection} \PY{k}{import} \PY{n}{f\PYZus{}regression}
          \PY{n}{f\PYZus{}regression}\PY{p}{(}\PY{n}{X\PYZus{}train}\PY{p}{,} \PY{n}{y\PYZus{}train}\PY{p}{,} \PY{n}{center}\PY{o}{=}\PY{k+kc}{True}\PY{p}{)}
\end{Verbatim}


\begin{Verbatim}[commandchars=\\\{\}]
{\color{outcolor}Out[{\color{outcolor}166}]:} (array([312.14499437,  98.42158757,  10.88729908]),
           array([1.46738970e-42, 4.35496600e-19, 1.14819587e-03]))
\end{Verbatim}
            

    % Add a bibliography block to the postdoc
    
    
    
    \end{document}
